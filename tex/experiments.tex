% !TEX root = ../thesis.tex
% !TEX spellcheck = en-US

%% Leave first page empty
\thispagestyle{empty}

\section{Experiments}
\label{sec:Experiments}

\todo{Doc2Vec model is evaluated in 2 ways (normal and trained on inferred vectors)}

With the final problem definition and dataset in place experiments were set up to evaluate the performance of the different methods to approach this problem that that are explained in Section~\ref{sec:Methods}. In this section the research objectives of these experiments will be defined and the metrics used will be described. Afterwards the experimental setup for each of the methods in the same order as the results for each approach will be presented in Section~\ref{sec:Results}.

\subsection{Objectives and Metrics}
\label{sub:Objectives and Metrics}

The experiments had simple objectives: The goal was to evaluate each method in terms of its effectiveness using a prediction metric well-suited for this problem. Further additional considerations, especially algorithmic time and space complexity were taken into account by comparing methods with regards to their runtime and need for computational resources.

\glsreset{MCC}

As a metric for predictive performance \gls{MCC} was chosen. This metric is used relatively rarely used as opposed to e.g.\ the F1 score that is common in the \gls{IR} literature where ignoring True Negatives can be tolerated. As an example we do not generally care if a search engine predicts correctly all the billions of website we don't want to see for a search query as long as it retrieves enough relevant ones.
With the dataset at hand though the needs were different and a metric was needed that can reliably and without bias measure prediction despite a strongly skewed distribution of labels, and stratified sampling to achieve a balanced distribution was not an option since the dataset was too small. \gls{MCC} fulfills these criteria as Section~\ref{sub:Evaluation Metrics for Text Classification} points out and additionally is easy to interpret as it is a correlation score between 1 (absolute correlation) and -1 (absolute anti-correlation) as opposed to e.g.\ the categorical cross-entropy error. In some experiments additionally further metrics were measured.

\subsection{Baseline}
\label{sub:Baseline (Experiments)}

As a baseline for comparing the performance of classification two different guessing strategies were used, namely uniform and stratified guessing.
Uniform guessing refers to a predictor that samples from the given classes assuming a uniform distribution whereas stratified guessing takes the label distribution in the data as the underlying probability distribution.
Then both methods just sample from these distributions to produce ``predictions'', while ignoring the actual input data.


\subsection{Approach 1: Vector Space Models}
\label{sub:Approach 1: Vector Space Models (Experiments)}

A popular way to approach text classification and other tasks in natural language processing is to build a model that maps data into a vector space so that distance between data points in this space translates to similarity between the objects (see Section~\ref{sub:Vector Space Models}). The resulting vector representation of the data can then be fed into various learning algorithms. This section describes the experiments performed to study this approach to the sentence classification problem as defined in Section~\ref{subs:Problem Formalism: Multi-class Sentence Classification}.

First the different approaches to produce such vector representations are compared. Several methods were used to generate vectors from the data while limiting dimensionality to 300 --- a heuristically chosen value as performance for the models did not increase significantly after beyond it. These vectors were then compared in terms of performance using them as input to the very simple classification method Logistic Regression.

Second, the most well-performing candidates to produce vector representations were chosen and a set of more advanced classification techniques was applied to them.

\todo{baseline comparison missing in later experiments?}

\subsubsection*{Evaluation of Vector Space Models}
\label{subs:Evaluation of Vector Space Models}

\paragraph{N-gram Models}
\label{par:N-gram Models (Experiments)}

The first class of language models that was investigated for the task of multi-class classification are N-gram models that were explained in Section~\ref{subs:n-gram-language-models}. As mentioned, in essence this type of model relies on simple statistics which makes for straightforward computation but at the same time comes at cost of expressiveness, especially in terms of temporal dependencies between words.

As N-grams models come in a variety of variants the most important ones were used as hyper-parameters to the model and a grid search was carried out over a wide range of combinations over these. The specific hyper-parameter settings are listed in Table~\ref{tab:N-gram Hyper-parameters Space}. The grid search was optimized with regards to \emph{Matthews Correlation Coefficient} (see Section~\ref{par:Matthews Correlation Coefficient for K classes}) using 5-fold cross-validated with three standard classifiers: Logistic Regression and Naive Bayes and SVM.

\begin{table}[h]
  \begin{center}
  \begin{tabular}{ l l l}
    \toprule
    Hyper-Parameter & N-gram Type: Words & N-gram Type: Characters \\
    \midrule
    N-gram Range (Range) & [1,1], [1,2], [1,3], [2,3], [3,3] & [1,5], [1,10], [5,10], [5,15] \\
    Stop Words & English, None & N/A \\
    Vector Size (Size) & 10, 100, 300 & 10, 100, 300 \\
    IDF & Yes, No & Yes, No \\
    Norm & L1, L2, None & L1, L2, None \\
    Sub-linear TF & Yes, No & Yes, No \\
    \bottomrule
  \end{tabular}
  \caption{Parameter search space word and character level N-gram models}
\label{tab:N-gram Hyper-parameters Space}
\end{center}
\end{table}

\paragraph{Bag-of-Means: An Averaged Word2Vec Model}
\label{par:Bag-of-Means: An Averaged Word2Vec Model}

Next a Bag-of-Means model as described in Section~\ref{subp:Bag-of-Means} was evaluated with the same set of classifiers. The model was evaluated on the same test and training data split as used for the N-gram model above. As a basis the pre-trained word-vectors from the Google News dataset\footnote{The dataset contains contains 300-dimensional vectors for 3 million words and phrases. The phrases were obtained using a simple data-driven approach described in~\cite{Mikolov:2013ab}. The dataset can be obtained on the following website: \url{https://code.google.com/archive/p/word2vec/}} were used and then for each document all word vectors were average to obtain the document vector.

\paragraph{Paragraph Vectors using Distributed Representations}
\label{par:Paragraph Vectors using Distributed Representations}

Next a vector space model was build using the approach proposed by~\cite{Le:2014aa} and described in more detail in Section~\ref{par:Distributed representations for documents}. Again there are several hyper-parameters to this model that are described in Section~\ref{subs:Language Models using Distributed Representations} and turn out to have a huge influence on its performance as the results below indicate. As this model is computationally quite expensive a grid search as for the N-gram model above was infeasible. Thus the effect of the hyper-parameters was studied by just varying them one at a time while keeping the others fixed, using a Logistic Regression classifier with 5-fold cross-validation.

\subsubsection*{Evaluation of Classification Methods}
\label{subs:Evaluation of Classification Methods}

\paragraph{Logistic Regression}
\label{par:Logistic Regression}

\paragraph{Decision Tree}
\label{par:Decision Tree}

\paragraph{Naive Bayes}
\label{par:Naive Bayes}

\paragraph{Support Vector Machine}
\label{par:Support Vector Machine}

\paragraph{$k$ Nearest Neighbors}
\label{par:k Nearest Neighbors}

\paragraph{Random Forest}
\label{par:Random Forest}

\paragraph{Vanilla Neural Network}
\label{par:Vanilla Neural Network}

\paragraph{Deep Neural Network}
\label{par:Deep Neural Network}

\paragraph{Convolutional Neural Network}
\label{par:Convolutional Neural Network}


\subsection{Approach 2: Sequential Models}
\label{sub:Approach 2: Sequential Models (Experiments)}
