% !TEX root = ../thesis.tex
% !TEX spellcheck = en-US

%% Leave first page empty
\thispagestyle{empty}

\section{Introduction}

Language is one of the most complex behaviors our species has developed. Humans use it to communicate even the most abstract concepts and it is considered one of the pillars of modern civilization. It takes children years to learn to communicate their thoughts and the subtle nuances of one's language give a glimpse one's cultural environment and upbringing, one's emotional state and one's intellect.

Not surprisingly in the field of Artificial Intelligence (AI) building computer systems with linguistic capabilities and solving language-based problems poses one of the hardest challenges and has motivated decades of research in Computational Linguistics. In fact many of the famous test for universal machine intelligence are based on linguistic capabilities, among them the famous \emph{Turing test} by~\cite{Turing:1950aa} where the task is for a human judge to determine whether he is having a conversation with a human or a machine in order to determine if the machine can be called intelligent, or the \emph{compression test} proposed by~\cite{Mahoney:1999aa}, where a human's and machine's capability to predict missing words given a context is tested.

This thesis explores the specific task of predicting the semantic structure of job advertisements as a specific example of such a language-based task that turns out to be difficult even for humans to do.
The work was done in close collaboration with the Helsinki-based media and learning company \emph{Sanoma}\footnote{\textquote{Sanoma is a front running consumer media and learning company in Europe. In Finland and the Netherlands we are the market leading media company with a broad presence across multiple platforms. In Belgium we are among the Top 5. Our main markets in learning are Belgium, Finland, the Netherlands, Poland and Sweden. We entertain, inform, educate and inspire millions of people every day. We employ some 7,500 professional employees operating in Europe.}, Source: \url{http://www.sanoma.com/en/who-we-are}, visited 06.06.2016} and the research motivation was thus constantly tied back into real world challenges in the scope of Sanoma's business needs.

% - what (problem statement)
% - why (need statement)
% - how (approach)
% - related
% - results

\subsection{Need Statement and Motivation}

Today's media and education, the basis of Sanoma's core businesses, are undergoing drastic and fundamental transformations that are currently disrupting whole industries.

Usage of digital media as a source of information has long surpassed print media. Sanoma's most well-known product, Finland's biggest daily newspaper \emph{Helsingin Sanomat}, lost 6\% of its circulation only in 2015\footnote{Source: http://www.digitalnewsreport.org/survey/2016/finland-2016/, visited 27.07.2016}, while the wide-spread use of social media challenges traditional ways we access information. Similarly in the field of education, with the rise of Massive open online course (MOOCs), traditional learning settings are challenged and the need for advanced techniques for data processing and analysis increases, e.g.\ to personalize and adapt the learning experience to each individual user and at the same time identify trends across large groups of learners to better meet the needs of education.

Sanoma provides a recruitment platform named \emph{Oikotie Työpaikat}. The service is in direct competition several other international players in the recruitment industry. Through this and other services Sanoma's collects large amounts of user-generated data, offering the potential to be leveraged for machine learning solutions to provide value for their users and innovate and enrich the company's offerings.
This was the company's initial motivation for this thesis project --- To explore ways to leverage user-generated data to potentially.

From my perspective this offered many interesting research possibilities while at the same time being relevant for a real business. Natural Language Processing and Computer Linguistics had always been of strong interest to me for the complex nature and yet high interpretability of problems and their proximity and relatedness to progress in universal machine intelligence. This presented me with the challenge position to balance pursuing research objectives and yet exploring potential business and user needs, learn on new fronts and deepen my knowledge in others as well as combining my experience in Product Innovation, thus made for a great project to mark the competition of my studies as a Master's student.

\subsection{Problem Statement}

The problem addressed during with this thesis to better understand the structure of job advertisements. In particular job postings typically consist of several parts with a certain function or theme: Usually company is introduced, the job is described with it's tasks and responsibilities, the requirements for the job are listed, then benefits and offerings by the company are named and the reader is asked to apply in a specified way he or she is interested.
Almost all of the text\footnote{Only 4\% of the sentences collected for evaluating the final experiments in this thesis were sorted into the category \emph{other} while the rest falls into either of the categories described. This is described in more detail in Section }\todo{link section} of a job description falls into these categories and the task can thus be posed as predicting a category for each sentence in a job advertisement, that corresponds with this sentence belonging to one of the job ads' parts as described above. This is a challenging problem in itself but can further be used to extract certain functional parts of each job ad, to study a possible correlation between structural patterns and the reach and success of an ad and so forth. The problem therefore can be labelled as \emph{text categorization} or \emph{text classification} as referred to in the scientific literature.

\subsection{Research Objectives and Scope}

- study vector space models and new approaches
- study sequential and multi-task approaches
- mu


\subsection{Related work *}

 % \emph{information retrieval} (IR)  - 90's

Algorithmic text categorization into a fixed set of categories has been of a topic of interest for decades.

 Boosted by the increasingly vast amounts of data available today. The applications are various, from document filtering, automated metadata generation such as language classification to automatic email labeling, spam identification and sentiment detection, amongst others.

Unsupervised techniques for topic discovery have been investigated widely, such as LSA

Vector Space models are a

- feature learning for text
- multitask learning


\cite{Collobert:2008aa} showed how both multitask learning and semi-supervised learning improve the generalization of the shared tasks on text data. They describe \textquote{a single convolutional neural network architecture that, given a sentence, outputs [\ldots] part-of-speech tags, chunks, named entity tags, semantic roles, semantically similar words and the likelihood that the sentence makes sense (grammatically and semantically) using a language model}.

\cite{Lodhi:2002aa} string kernels

\todo{section on transfer learning and feature learning}
\todo{text classification}
\todo{Multitask learning}
\todo{explicit vs implicit feature representation}

\subsection{Structure of the thesis*}

The first section of this thesis gave a brief introduction into the topic of this work, outlined the motivation and the research problem approached and showed the research objectives, the scope of the thesis as well as related work.

Section~\ref{sec:Background}:~\nameref{sec:Background} introduces the reader to concepts and ideas of Text Classification in order to provide him or her with the necessary knowledge to understand the work described in this thesis. First the problem of Text Classification is formally defined and the most common approaches to this problem are described on a high level. Afterwards Vector Space Models are introduced in detail, which represent a popular way of tackling this task by transforming text into fixed-size vectors.
The following subsection then briefly presents several of the most known classification algorithms that can operate on the vectors produced by such Vector Space Models. Next the approach of Sequential Classification is described in short where text is treaded as a one-dimensional signal in time. The following subsection then shows common ways to evaluate how well the task of text classification is solved and the last subsection gives a brief introduction to two methods that are useful for exploring different models and techniques through visualization.

Section~\ref{sec:Exploration}:~\nameref{sec:Exploration} gives an overview of the exploration of the wider topic space at the start of the thesis project as well as the experimentation with different techniques and the data given for the project. It aims to lay out to the reader the process, insights and learnings leading towards the final problem formulation and evaluation of methods to solve this problem.

The following Section~\ref{sec:Multi-class Prediction of Semantic Categories for Text in Job Advertisements}:~\nameref{sec:Multi-class Prediction of Semantic Categories for Text in Job Advertisements} then presents the main results of the thesis. The exact problem definition is given that is used as a basis to evaluate the experiments and the dataset used for these experiments is described. Subsequently the evaluation of the different approaches to Vector Space Models, the classification algorithms using the most successful of these models and the sequential modeling approach are documented. Each of these subsections first describes the experimental setup, followed by results of the different approaches and closes with a brief discussion on the results presented. Then
