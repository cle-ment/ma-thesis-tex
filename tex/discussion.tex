% !TEX root = ../thesis.tex
% !TEX spellcheck = en-US

\clearpage

\section{Discussion}
\label{sec:Discussion}


\subsection{Vector Space Models}
\label{sub:Vector Space Models (Discussion)}

\subsubsection{N-gram Models}

Not without reason are N-Gram Models still so widely used as the experiments in Section~\ref{sec:Experimental Results} confirm. They are easy to understand, straightforward to train using publicly available software libraries, fast to compute and yield good performance for many problems in \gls{NLP}.

Hyper-parameters have to be adjusted before training the model as they affect the results differently given the problem and also the used classifier. As we saw in Section~\ref{subs:N-gram Models (Experiments and Results)} the type of N-gram (word vs.\ character-based) and the vector dimensionality have the strongest impact on the results. Other hyper-parameters do not have a substantial influence, for instance stop-words, or do not seem to matter at all in this particular settings such as Sub-linear term frequency scaling.

Optimizing for hyper-parameters though is very cheap as well as an N-gram model is so fast to train, allowing to easily and automatically explore a wide parameter-space with simple \gls{Grid Search}.

N-Grams thus make for a great choice as a baseline model with good performance and simply usage.

\subsubsection{Bag-of-Means Model}

Composition of feature vectors to produce a feature representation for a group of objects is a known approach (see e.g.~\cite{Mitchell:2010aa} for a study on compositions of word embeddings). However this technique is also seen as an ad-hoc solution with unforeseeable side-effects such as flattening out the significance of complementary patterns as the number of vectors increase. As~\cite{Zhang:2015aa} puts it: \textquote{Bag-of-means is a misuse of word2vec [...]} and \textquote{such a simple use of a distributed word representation may not give us an advantage to text classification}. Further~\cite{Le:2014aa} state that \textquote{weighted averaging of word vectors [...] loses the word order in the same way as the standard bag-of-words models do}.

Given these previous results and thoughts it is the more surprising how well the Bag-of-Means model performs, achieving the highest performance of all models in combination with a simple \gls{Neural Network}. This type of model might work well in this specific case for two reasons:
First the word vectors that were used had been trained on a humongous dataset, implicit knowledge that the model could leverage while the other vector space models could not. Second the dataset consists of sentences, so short sequences of text where the problem of vectors negating each other's patterns can be expected to not be as severe.

Regarding computational complexity this model is relatively fast to train if one uses pre-computed word-vectors as done in the experiments. However the training of such corpus of 1 billion word-vectors itself takes enormous time and resources.

\subsubsection{Paragraph Vectors using Distributed Representations}

The Paragraph Vector model is appealing with it's inspiration from word embeddings and has been shown to outperform other text representations, specifically N-gram models (see e.g.~\cite{Le:2014aa}). However significant effort is needed to achieve good results with this technique and it is especially sensitive to the configuration of its hyper-parameters which are challenging to extensively test due to the computational resources needed.

With regards to the hyper-parameters, the choice the biggest impact on performance is the chosen model architecture. The results on this specific task and dataset show that contrary to the observations in~\cite{Le:2014aa} the Distributed Bag-Of-Words (PV-DBOW) architecture achieves significantly higher performance (an absolute difference of almost 14\% in terms of \gls{MCC} test score) and is more stable towards overfitting.
As in other vector space models the dimensionality of the vector representation naturally has a strong impact on performance, however the with decreasing vector size performance does not suffer as much as N-Grams models do and the model still achieves an \gls{MCC} test score of 0.223 with only 2 dimensions. This corresponds to a relative loss in performance by a factor of 2/3 while scaling down in dimensionality by a factor of 150. Further adding to the complexity of finding a good model several hyper-parameter interact with each other. For instance sub-sampling of frequent words only helped performance when using the Distributed Memory (PV-DM) model and otherwise significantly harmed it.

An interesting observation is that a number of classifiers tend to overfit using the vectors produced by this model. This could mean that it produces clearer patterns in the representation of documents and a richer representation. It is very likely that this type of model needs a much larger dataset in order to achieve it's highest performance as for many models based on Neural Network architectures. As mentioned earlier the word2vec models in~\cite{Mikolov:2013ad} were trained on a dataset of a billion words.

\subsubsection{Comparison of Vector Space Models}
\label{subs:Comparison of Vector Space Models}

- computational resources

- feature engineering vs from scratch



As mentioned earlier the computational needs to train this model are higher than an N-Grams model by an order of magnitudes.

- Ngrams very feature engineering




% MORE DATA needed
- comparision between only training set with validation split and traning + test for vector space models with Log Reg, 10\% lower performance almost


\subsubsection{Classification Methods}
\label{subs:Classification Methods}


\subsection{Sequential Modeling}
\label{sub:Sequential Modeling}



\subsection{Comparison of Approaches}
