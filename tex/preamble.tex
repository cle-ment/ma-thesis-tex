% !TEX root = ../thesis.tex
% !TEX spellcheck = en-US

\university{Aalto University}
\school{School of Science}

\department{Department of Information and Computer Science}
\professorship{Machine Learning, Data Mining, and Probabilistic Modeling}

\univdegree{MSc}

\author{Clemens Westrup}

%% Your thesis title comes here and again before a possible abstract.
%% If the title is very long and latex does an
%% unsatisfactory job of breaking the lines, you will have to force a
%% linebreak with the \\ control character.
%% Do not hyphenate titles.
%%
\thesistitle{An Exploration of Representation Learning and Sequential Modeling Approaches for Supervised Topic Classification in Job Advertisements}

\place{Helsinki}

\date{25.09.2016}

%% B.Sc. or M.Sc. thesis supervisor
%% Note the "\" after the comma. This forces the following space to be
%% a normal interword space, not the space that starts a new sentence.
%% This is done because the fullstop isn't the end of the sentence that
%% should be followed by a slightly longer space but is to be followed
%% by a regular space.
%%
\supervisor{Prof.\ Aristides Gionis}

%% B.Sc. or M.Sc. thesis advisors(s). You can give upto two advisors in
%% this template. Check with your supervisor how many official advisors
%% you can have.
%%
% \advisor{Clemens Westrup}
\advisor{Ph.D.\ Michael Mathioudakis}
% \advisor{M.Sc.\ Mikko Takkunen}

%% Aalto logo: syntax:
%% \uselogo{aaltoRed|aaltoBlue|aaltoYellow|aaltoGray|aaltoGrayScale}{?|!|''}
%%
%% Logo language is set to be the same as the document language.
\uselogo{aaltoYellow}{!}

%% Create the coverpage
%%
\makecoverpage

%% English abstract.
%% All the information required in the abstract (your name, thesis title, etc.)
%% is used as specified above.
%% Specify keywords
%%
%%
\keywords{NLP, bla bla, keyword}
%% Abstract text
\begin{abstractpage}[english]

This thesis applies the explorative double diamond design process borrowed to iteratively frame a research problem applicable in the context of a recruitment web service and then find the best approach to solve it. Thereby the problem focus is laid on multi-class classification, in particular the task of labelling sentences in job advertisements with one of six topics which were found to be covered in every typical job description. A dataset is obtained for evaluation and conventional N-Gram Vector Space models are compared with Representation Learning approaches, notably continuous distributed representations, and Sequential Modeling techniques using Recurrent Neural Networks. Results of the experiments show that the Representation Learning and Sequential Modeling approaches perform on par or better than traditional feature engineering methods and show a promising direction in and beyond research in Computational Linguistics and Natural Language Processing.

\end{abstractpage}

%% Force a new page so that the possible English abstract starts on a new page
\newpage


%% Preface
\mysection{Preface}


%
% one has to find a proper balance between the tendency towards perfectionism and

% - Michael: for all the incredible support, patience, interest and caring
% - Aris: for providing me with this amazing opportunity to do this
% - Juho: for introducing me to proper scientific work
% - Henning: for discussions, a great friendship
% - Martti: for being an awesome friend
% - my family: for something something
% - Mikko
% - Mika
% - Sami and Hugo
% - the design factory family

% - foundation of aalto
% - Sanoma for funds
% - Aris for funding my crowd-sourcing

I want to thank some people \ldots
\\

\vspace{5cm}
Helsinki, 25.09.2016

\vspace{5mm}
{\hfill Clemens Westrup \hspace{1cm}}

%% Force new page after preface
\newpage

%% Table of contents.
{\hypersetup{linkcolor=black}
% or \hypersetup{linkcolor=black}, if the colorlinks=true option of hyperref is used
\thesistableofcontents
}
\clearpage

\mysection{Notation}

\subsection*{Numbers and Arrays}

\begin{tabular}{ll}
  $a$                 & A scalar (integer or real) \\
  $\mathbi{a}$        & A vector \\
  $\mathbi{A}$        & A matrix \\
\end{tabular}

\subsection*{Sets and Graphs}

\begin{tabular}{ll}
  $\mathbb{A}$        & A set \\
  $\mathbb{R}$        & The set of real numbers \\
  $\{0, 1\}$          & The set containing 0 and 1 \\
  $\{0,1,\ldots,n\}$  & The set of all integers between 0 and n \\
  $[a, b]$            & The real interval including a and b \\
  $(a, b]$            & The real interval excluding a but including b \\
\end{tabular}



%% Tweaks the page numbering to meet the requirement of the thesis format:
%% Begin the pagenumbering in Arabian numerals (and leave the first page
%% of the text body empty, see \thispagestyle{empty} below).
%% Additionally, force the actual text to begin on a new page with the
%% \clearpage command.
%% \clearpage is similar to \newpage, but it also flushes the floats (figures
%% and tables).
%% There is no need to change these
%%
\cleardoublepage
\storeinipagenumber
\pagenumbering{arabic}
\setcounter{page}{1}
