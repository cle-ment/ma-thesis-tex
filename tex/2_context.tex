% !TEX root = ../thesis.tex
% !TEX spellcheck = en-US

\clearpage
\section{Context *}

\subsection{Background *}



\subsection{Corporate Partner}

This thesis was done in close and inspiring collaboration with the Helsinki-based company \emph{Sanoma}. The company describes itself as follows\footnote{Source: \url{http://www.sanoma.com/en/who-we-are}, visited 06.06.2016}:

\blockquote{Sanoma is a front running consumer media and learning company in Europe. In Finland and the Netherlands we are the market leading media company with a broad presence across multiple platforms. In Belgium we are among the Top 5. Our main markets in learning are Belgium, Finland, the Netherlands, Poland and Sweden. We entertain, inform, educate and inspire millions of people every day. We employ some 7,500 professional employees operating in Europe.}

\subsection{Need Statement}

Today's media and education, Sanoma's core businesses, are undergoing drastic and fundamental transformations that are currently disrupting whole industries. \todo{some numbers here?}

Usage of digital media as a source of information has long surpassed print media \todo{source, e.g. http://www.journalism.org/2015/04/29/state-of-the-news-media-2015/} and the wide-spread use of social media challenges traditional ways we access information. \todo{example, e.g. fb usage compared to newspaper}

Similarly, with the rise of Massive open online course, so-called MOOCs, traditional learning settings are challenged and increase the need for advanced techniques for data processing and analysis, e.g. to personalize and adapt the learning experience to each individual user and at the same time identify trends across large groups of learners to better meet the needs of education.

Sanoma provides a recruitment platform named \emph{Oikotie Työpaikat}. The service is in direct competition several other international players in the recruitment industry.

\subsection{Problem Statement *}



\subsection{Research objectives *}

\subsection{Related work *-}

 % \emph{information retrieval} (IR)  - 90's

Algorithmic \emph{text categorization} (TC --- also known as \emph{text classification}) into a fixed set of categories has been of a topic of growing interest during the last decades, boosted by the increasingly vast amounts of data available today. The applications are various, from document filtering, automated metadata generation such as language classification to automatic email labeling, spam identification and sentiment detection, amongst others.

Unsupervised techniques for topic discovery have been investigated widely, such as LSA

Vector Space models are a

- feature learning for text
- multitask learning


\cite{Collobert:2008aa} showed how both multitask learning and semi-supervised learning improve the generalization of the shared tasks on text data. They describe \textquote{a single convolutional neural network architecture that, given a sentence, outputs [\ldots] part-of-speech tags, chunks, named entity tags, semantic roles, semantically similar words and the likelihood that the sentence makes sense (grammatically and semantically) using a language model}.

\cite{Lodhi:2002aa} string kernels

\todo{section on transfer learning and feature learning}
\todo{text classification}
\todo{Multitask learning}
\todo{explicit vs implicit feature representation}
