% !TEX root = ../thesis.tex
% !TEX spellcheck = en-US

\newglossaryentry{api}{type=\acronymtype, name={API}, description={Application
Programming Interface}, first={Application
Programming Interface (API)\glsadd{apig}}, see=[Glossary:]{apig}}

\newglossaryentry{apig}{name={API},
    description={An Application Programming Interface (API) is a particular set of rules and specifications that a software program can follow to access and make use of the services and resources provided by another particular software program that implements that API}}

\newglossaryentry{knn}{type=\acronymtype, name={kNN}, description={k-Nearest Neighbors}, first={k-Nearest Neighbors (kNN)\glsadd{knng}}, see=[Glossary:]{knng}}

\newglossaryentry{knng}{name={$k$-Nearest Neighbors},
    description={The $k$-Nearest Neighbors algorithm is a non-parametric method for classification or regression tasks. See Section~\ref{par:K Nearest Neighbors}}}

\newglossaryentry{ann}{type=\acronymtype, name={ANN}, description={Artificial Neural Network}, plural={Artificial Neural Networks}, first={Artificial Neural Network (ANN)\glsadd{anng}}, see=[Glossary:]{anng}}

\newglossaryentry{anng}{name={Artificial Neural Networks (ANN)},
description={or simply Neural Networks (NN) are a learning algorithm inspired by the \todo{finish}. See Section~\ref{par:K Nearest Neighbors}}}

\newglossaryentry{NLTK}{type=\acronymtype, name={NLTK}, description={Natural Language Toolkit}, first={Natural Language Toolkit (NLTK)\glsadd{Natural Language Toolkit}}, see=[Glossary:]{Natural Language Toolkit}}

\newglossaryentry{Natural Language Toolkit}{name={Natural Language Toolkit (NLTK)}, description={``NLTK is a leading platform for building Python programs to work with human language data. It provides easy-to-use interfaces to over 50 corpora and lexical resources such as WordNet, along with a suite of text processing libraries for classification, tokenization, stemming, tagging, parsing, and semantic reasoning and wrappers for industrial-strength NLP libraries.'' See \url{http://www.nltk.org}}}

\newglossaryentry{langdetect}{name={langdetect},
    description={Language detection library ported from Google's language-detection (\url{https://code.google.com/p/language-detection/}). See \url{https://pypi.python.org/pypi/langdetect?}}}

\newglossaryentry{Beautiful Soup}{name={Beautiful Soup},
    description={Beautiful Soup is a Python library designed for quick turnaround projects like screen-scraping. See \url{https://www.crummy.com/software/BeautifulSoup/}}}

\newglossaryentry{confusion matrix}{name={Confusion Matrix},
    description={todo}}

\newglossaryentry{Oikotie Tyopaikat}{name={Oikotie Työpaikat},
    description={todo}}

\newglossaryentry{Sanoma}{name={Sanoma},
    description={todo}}

\newglossaryentry{error matrix}{name={Error Matrix},
    description={todo},see=[Glossary:]{confusion matrix}}

\newglossaryentry{ffe}{type=\acronymtype, name={FFE}, description={Fuzzy Front End}, first={Fuzzy Front End (FFE)\glsadd{ffeg}}, see=[Glossary:]{ffeg}}

\newglossaryentry{ffeg}{name={Fuzzy Front End},
    description={The early exploration phase in the design process (see~\cite{Council:2007aa}). Also refers to an awesome band from Finland.}}

\newglossaryentry{Ensemble Method}{name={Ensemble Method}, plural={Ensemble Methods}, description={TODO}}

\newglossaryentry{Randomized Algorithm}{name={Randomized Algorithm}, plural={Randomized Algorithms}, description={TODO}}

\newglossaryentry{Boosting}{name={Boosting}, description={TODO}}

\newglossaryentry{Bagging}{name={Bagging}, description={TODO}}

\newglossaryentry{Voting}{name={Voting}, description={TODO}}

\newglossaryentry{Feature Selection}{name={Feature Selection}, description={TODO}}

\newglossaryentry{Feature Extraction}{name={Feature Extraction}, description={TODO}}

\newglossaryentry{Principal Components Analysis}{name={Principal Components Analysis}, description={TODO}}

\newglossaryentry{PCA}{type=\acronymtype, name={PCA}, description={Principal Components Analysis}, first={Principal Components Analysis (PCA)\glsadd{Principal Components Analysis}}, see=[Glossary:]{Principal Components Analysis}}

\newglossaryentry{t-Distributed Stochastic Neighbor Embedding}{name={t-Distributed Stochastic Neighbor Embedding}, description={t-Distributed Stochastic Neighbor Embedding (t-SNE) is a (prize-winning) technique for dimensionality reduction that is particularly well suited for the visualization of high-dimensional datasets. For more information visit the authors information website (\url{https://lvdmaaten.github.io/tsne/}) and see the introductory publication (\cite{Maaten:2008aa})}}

\newglossaryentry{t-SNE}{type=\acronymtype, name={t-SNE}, description={t-Distributed Stochastic Neighbor}, first={t-Distributed Stochastic Neighbor (t-SNE)\glsadd{t-Distributed Stochastic Neighbor Embedding}}, see=[Glossary:]{t-Distributed Stochastic Neighbor Embedding}}

\newglossaryentry{Dimensionality Reduction}{name={Dimensionality Reduction}, description={TODO}}

\newglossaryentry{svm}{type=\acronymtype, name={SVM}, description={Support Vector Machine}, first={Support Vector Machine (SVM)\glsadd{Support Vector Machine}}, see=[Glossary:]{Support Vector Machine}}

\newglossaryentry{Support Vector Machine}{name={Support Vector Machine},
    description={TODO}}

\newglossaryentry{CrowdFlower}{name={CrowdFlower},
    description={TODO}}

\newglossaryentry{crowd sourcing}{name={crowd sourcing},
    description={TODO}}

% SVM & Support Vector Machine \\
% NN & Neural Network \\
% RNN & Recurrent Neural Network \\
% LSTM & Long Short-Term Memory \\
% MCC & Matthews Correlation Coefficient, see Section~\ref{par:Informedness, Markedness and Matthews Correlation Coefficient}


% Bayes' Theorem
% one-hot-encoding
% grid search
% Mturk & see \emph{Mechanical Turk} \\
% Mechanical Turk
% API
% MongoDB
% Mongoose
% GitHub
% decision boundary
% discriminant function
% gradient descent
% Bias-Variance Dilemma
